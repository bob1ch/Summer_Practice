\documentclass[../document.tex]{subfiles}

\begin{document}
	\par По полученным результатам исследования можно сделать вывод, что модель yolov9e является наиболее точной в сравнении с другими исследованными моделями. Также эта модель имеет наибольшую площадь под кривой ROC. Изучив матрицу ошибок по этой модели можно обратить внимание, что у модели не было ложно-отрицательных срабатываний. Это означает, что модель не пропустила ни одного изображения содержащего медведя, и в случае когда речь идет о жизнях людей это имеет решающий фактор. Неплохой результат показали RT-DETR, которые уступили лишь одной модели в точности.
	\par Также модели исследовались на синтетических данных в качестве тестовых. Это была созданная сцена с медведем в лесу. Основная проблема с которой  столкнулись исследуемые модели заключалась в том, что не получалось обнаружить медведя на этих изображеиях. Эту проблему удавалось нивелировать дообучением исследованных моделей на синтетических данных, что могло бы помочь в задаче детекции медведей в специфической местности.
    \par Во время прохождения практики были успешно выполнены все поставленные
 задачи. Получены первичные профессиональные умения и навыки в области разработки программных продуктов с применением современных информационных технологий, сформировано единство теоретической и практической подготовки, закреплены теоретические знания и практические умения, был получен опыт самостоятельной сборки, анализа и проверки внешней и внутренней информации, сформированы навыки творческого решения проблем проектной и производственно-технологической деятельности.
    \par При выполнении индивидуального практического задания со стороны руководителя практики и преподавателей была оказана значительная помощь в поиске и изучении нового материала, выполнении заданий, оформлении отчёта.
\end{document}